\documentclass{article}

% Language setting
% Replace `english' with e.g. `spanish' to change the document language
\usepackage[english]{babel}
\usepackage{mathtools}
\usepackage{amssymb}

% Replace `letterpaper' with `a4paper' for UK/EU standard size
\usepackage[letterpaper,top=2cm,bottom=2cm,left=3cm,right=3cm,marginparwidth=1.75cm]{geometry}
\usepackage{listings}

\DeclarePairedDelimiter\ceil{\lceil}{\rceil}
\DeclarePairedDelimiter\floor{\lfloor}{\rfloor}

% Useful packages
\usepackage{amsmath}
\usepackage{graphicx}
\usepackage[colorlinks=true, allcolors=blue]{hyperref}

\title{FilterFlow - transformation reference}
\author{Dominik Lau, Łukasz Głazik, Tadeusz Brzeski}

\begin{document}
\maketitle
\subsubsection*{Nomenclature}
$In_i$ - $i$-th input of a transformation \\
$Out$ - ouput of a transformation \\
$I(x,y)$ - image's $I$ pixel value with coordinates $x, y$; $I(x,y) \in [0, 255]^3$, $I(x,y) = (r, g, b)^T$ \\
$K$ - kernel (for convolutions) \\
$size(K)$ - kernel size \\
$K(i, j)$ - kernel value with coordinates $(i, j)$, where $i,j\in 0..size(K)-1$ \\
$\oplus$ - XOR 
\section{Source}
\subsection{Source}
User-uploaded source
\subsection{Red}
Only red color i.e.
\begin{gather*}
    Out(x,y) = (1, 0, 0)^T
\end{gather*}
Image size is $512 \times 512$
\subsection{White noise}
White noise as output with a random seed. Image size is $512 \times 512$. Pattern changes on refresh.
\subsection{perlin noise}
Perlin noise as output with 5 octaves, persistence 0.5 and a random seed \cite{perlin}. Image size is $512 \times 512$. Pattern changes on refresh.
\section{Linear}
\subsection{Custom kernel}
A convolution with user-defined kernel. Available kernel sizes 2, 3 and 4.
Let $N = size(K)$. The convolution does the following
\begin{gather*}
    Out(x, y) = \sum_{i=0}^{N-1}\sum_{j=0}^{N-1} I(x + i - \floor*{N/2}, y + j - \floor*{N/2}) * K(i,j)
\end{gather*}
The pixel we take neighborhood for is determined by the above equation and it's the top left pixel for kernel size 2, middlemost for kernel size 3 and pixel with coordinates corresponding to $K(1,1)$.
\subsection{4-connected Laplace}
4-connected Laplace is like Custom kernel with kernel
\begin{gather*}
    K = \begin{bmatrix} 
    0 & 1 & 0 \\ 
    1 & -4 & 1 \\ 
    0 & 1 & 0
    \end{bmatrix}
\end{gather*}
\subsection{8-connected Laplace}
8-connected Laplace is like Custom kernel with kernel
\begin{gather*}
    K = \begin{bmatrix} 
    1 & 1 & 1 \\ 
    1 & -8 & 1 \\ 
    1 & 1 & 1
    \end{bmatrix}
\end{gather*}
\subsection{Gaussian}
Gaussian is like Custom kernel with kernel
\begin{gather*}
    K = \begin{bmatrix} 
    0.0625 & 0.125 & 0.0625 \\ 
    0.125 & 0.25 & 0.125 \\ 
    0.0625 & 0.125 & 0.0625
    \end{bmatrix}
\end{gather*}
\subsection{Sobel X}
Sobel X is like Custom kernel with kernel
\begin{gather*}
    K = \begin{bmatrix} 
    1 & 0 & 0 \\ 
    -2 & 0 & 0 \\ 
    1 & 0 & 0
    \end{bmatrix}
\end{gather*}
\subsection{Sobel Y}
Sobel Y is like Custom kernel with kernel
\begin{gather*}
    K = \begin{bmatrix} 
    1 & -2 & 1 \\ 
    0 & 0 & 0 \\ 
    0 & 0 & 0
    \end{bmatrix}
\end{gather*}
\section{Pooling}
Pooling takes a block of pixels of size $N \times N$ and 
calculates reduction functions on it. Then it moves $step$ pixels in one direction. The step value is (called stride) also customizable.
For instance, if the pooling size is $3$ and step is $4$ it is going to apply a reduction function on a block of size $3 \times 3$ and move 4 pixels to the next block of size $3 \times 3$. Each pooling changes the size of the output image
\begin{gather*}
    Out.width =\floor*{\frac{In.width -1}{stride + 1}}\\
    Out.height = \floor*{\frac{In.height -1}{stride + 1}}\\
\end{gather*}
Available strides: 1, 2, 3, 4. Available pooling sizes: 2, 3, 4.
\begin{gather*}
\end{gather*}
Let $N=$ pooling size. The pooling operation expressed mathematically
\begin{gather*}
    out(x,y) = \odot_{i=0..N-1} \odot_{i=0..N-1} I(x + i - \floor*{N/2}, y + j - \floor*{N/2})
\end{gather*}
where $\odot$ is the reduction function i.e. $\odot: \mathbb{R}^N \longrightarrow \mathbb{R}$
\subsection{Max pooling}
It is a pooling operation with $\odot = max$
\subsection{Min pooling}
It is a pooling operation with $\odot = min$
\subsection{Avg pooling}
It is a pooling operation with $\odot = average$
\section{Logical}
\subsection{And}
For a given argument $p$
\begin{gather*}
    Out(x,y) = p \land In(x, y)
\end{gather*}
\subsection{Or}
For a given argument $p$
\begin{gather*}
    Out(x,y) = p \lor In(x, y)
\end{gather*}
\subsection{Xor}
For a given argument $p$
\begin{gather*}
    Out(x,y) = p \oplus In(x, y)
\end{gather*}
\subsection{Binary And}
Binary as in two arguments.
\begin{gather*}
    Out(x, y) = In_1(x, y) \land In_2(x,y)
\end{gather*}
\subsection{Binary Or}
\begin{gather*}
    Out(x, y) = In_1(x, y) \lor In_2(x,y)
\end{gather*}
\subsection{Binary Xor}
\begin{gather*}
    Out(x, y) = In_1(x, y) \oplus In_2(x,y)
\end{gather*}
\section{Binary}
\subsection{Addition}
\begin{gather*}
    Out(x, y) = In_1(x, y) + In_2(x,y)
\end{gather*}
\subsection{Multiply}
\begin{gather*}
    Out(x, y) = In_1(x, y) * In_2(x,y)
\end{gather*}
\subsection{Substraction}
\begin{gather*}
    Out(x, y) = In_1(x, y) - In_2(x,y)
\end{gather*}
\section{Point}
\subsection{Brightness}
For a given argument $p$
\begin{gather*}
    Out(x,y) = pIn(x,y)
\end{gather*}
\subsection{Threshold}
For a given argument $p$
\begin{gather*}
    \begin{cases}
        Out(x,y) = 255 \text{      for $I(x,y) > p$} \\
        Out(x,y) = 0   \text{      for $I(x,y) \le p$}
    \end{cases}
\end{gather*}
\subsection{Grayscale}
\begin{gather*}
    Out(x,y) = In(x,y).r * 0.299 + In(x,y).g * 0.587 + In(x,y).b * 0.114;
\end{gather*}
\subsection{To YCbCr}
\begin{gather*}
    Out(x,y) = \begin{bmatrix} 
    0.299 & 0.587 & 0.114 \\ 
    -0.169 & -0.331 & 0.5 \\ 
    0.5 & -0.419 & -0.081
    \end{bmatrix} In(x,y) + (0, 127.5, 127.5)^T
\end{gather*}
\subsection{From YCbCr}
\begin{gather*}
    Out(x,y) = \begin{bmatrix} 
    1 & 0 & 1.4 \\ 
    1 & -0.343 & -0.711 \\ 
    1 & 1.765 & 0
    \end{bmatrix} (In(x,y) - (0, 127.5, 127.5)^T)
\end{gather*}
\subsection{R channel}
\begin{gather*}
    Out(x,y) = (In(x,y).r, 0, 0)^T
\end{gather*}
\subsection{G channel}
\begin{gather*}
    Out(x,y) = (0, In(x,y).g, 0)^T
\end{gather*}
\subsection{B channel}
\begin{gather*}
    Out(x,y) = (0, 0, In(x,y).b)^T
\end{gather*}
\subsection{Channel combination}
This transformation combines three images, it takes r channel of the first one, g channel of the second one and b channel of the third one and sets them as output channels
\begin{gather*}
    Out(x,y) = (In_1(x,y).r, In_2(x,y).g, In_3(x,y).b)^T
\end{gather*}
\section{Morphologic}
\subsection{Erosion}
For each neighborhood of pixels (that means for a kernel of $size(K) = N$) it performs a reduction together with logical and 
\begin{gather*}
    Out(x,y).r = \bigwedge_{i=0..N-1} \bigwedge_{j=0..N-1}{In(x+i,y+j).r > t} \\
    Out(x,y).g= \bigwedge_{i=0..N-1} \bigwedge_{j=0..N-1}{In(x+i,y+j).g > t} \\
    Out(x,y).b = \bigwedge_{i=0..N-1} \bigwedge_{j=0..N-1}{In(x+i,y+j).b > t}
\end{gather*}
where $t=0.5$.
\subsection{Dilatation}
For each neighborhood of pixels (that means for a kernel of $size(K) = N$) it performs a reduction together with logical and 
\begin{gather*}
    Out(x,y).r = \bigvee_{i=0..N-1} \bigvee_{j=0..N-1}{In(x+i,y+j).r > t} \\
    Out(x,y).g= \bigvee_{i=0..N-1} \bigvee_{j=0..N-1}{In(x+i,y+j).g > t} \\
    Out(x,y).b = \bigvee_{i=0..N-1} \bigvee_{j=0..N-1}{In(x+i,y+j).b > t}
\end{gather*}
where $t=0.5$.
\subsection{Skeletonization}
The image is skeletonized using the simplified algorithm\cite{skel}. Only 2 iterations (due to time complexity and us wanting to keep it real-time). Pseudocode of one iteration:
\begin{lstlisting}
    eroded = erodsion(img) 
    temp = dilation(eroded)
    temp = img - temp
    skel = bitwise_or(skel, temp)
    img = eroded
\end{lstlisting}
\section{Other}
\subsection{Mux}
A transformation that passes selected input forward. Let $i$ be the selected image index, then
\begin{gather*}
    Out(x,y) = In_i(x,y)
\end{gather*}

\bibliographystyle{plain}            
\begin{thebibliography}{3}                     
\bibitem{perlin}
https://adrianb.io/2014/08/09/perlinnoise.html
\bibitem{skel}
https://gist.github.com/jsheedy/3913ab49d344fac4d02bcc887ba4277d
\end{thebibliography}

\end{document}

